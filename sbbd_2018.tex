\documentclass[12pt]{article}

\usepackage{sbc-template}
\usepackage{graphicx,url}
\usepackage[utf8]{inputenc}
\usepackage[T1]{fontenc}
\usepackage[english, portuguese]{babel}
\usepackage{hyphenat}
\hyphenation{mate-mática recu-perar}
\usepackage{amsmath}
\usepackage{graphicx}
\graphicspath{ {imagens/} }
\usepackage{float}
\usepackage{environ}

\NewEnviron{myequation}{%
\begin{equation}
\scalebox{1.5}{$\BODY$}
\end{equation}
}

\sloppy

\begin{document} 

\title{Avaliando o Impacto da Modelagem de Dados em Aplicações OLAP Utilizando SGBD Relacional e Colunar}
\author{}
\address{}

\maketitle

\begin{abstract}
\end{abstract}
     
\begin{resumo}
\end{resumo}

\section{Introdução}
Para tomar decisões de maneira rápida e inteligente Organizações têm como base os dados armazenados em seus repositórios.
Conforme o volume destes dados aumenta a má estruturação do repositório pode degradar o desempenho do acesso às informações,
e devido a isso a tecnologia de Data Warehouse é utilizada para persistência de dados e tomada de decisões de forma eficiente.

Além do armazenamento de dados, é necessário que se tenha uma aplicação que os acesse e traga informações
de maneira tão eficiente quanto à estruturação do repositório. Aplicações OLAP analisam dados multidimensionalmente 
e executam consultas analíticas, fazendo com que estejam fortemente associadas à Data Warehouses.
Ao conjunto de Data Warehouse e aplicações OLAP denomina-se ambiente OLAP.

Um ambiente OLAP deve ser projetado e implantado visando a rapidez na recuperação de dados \cite{wrembel2007data}, 
\cite{codd1998providing}, \cite{kimball2002dw}. Sistemas Gerenciadores de Bancos de Dados (SGBD) surgem como uma solução 
para esta questão, e duas classes de SGBD podem atuar como gerenciadores de Data Warehouses, os relacionais e os NoSQL. 
Para que haja uma conclusão sobre qual abordagem é a melhor, a aplicação de um benchmark torna-se pertinente, 
e no escopo de banco de dados um benchmark que se destaca por tratar de sistemas de suporte à decisão é o TPC-H \cite{tpch2017page}. 

Outro fator que interfere no desempenho de um ambiente OLAP é a forma como este é modelado. 
Ambientes normalizados são utilizados pela facilidade de manutenção,
e o fácil entendimento acerca do relacionamento entre as entidades, que traz uma visão mais clara do sistema \cite{bax2003modelagem}. 
Entretanto, modelos denormalizados surgem para trazer ganho no desempenho das consultas, por diminuir 
as junções entre tabelas devido ao menor número de entidades.

Dadas estas condições, o objetivo aqui é realizar um estudo comparativo entre um SGBD relacional e outro NoSQL 
como gerenciadores de Data Warehouse utilizando o TPC-H. Devido a seu amplo uso e poucas modificações na linguagem SQL, foi definido o PostgreSQL 
como SGBD relacional e o MonetDB como NoSQL, este por ainda utilizar SQL, possuir interface simples e ser o pioneiro 
entre os bancos NoSQL colunares.

\section{Recuperação e Persistência de Dados}




\section{TPC-H}

\section{O Experimento}

\section{Conclusão}

\bibliographystyle{sbc}
\bibliography{REFERENCES}

\end{document}
